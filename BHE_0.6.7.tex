\documentclass[12pt, letterpaper]{article}
\usepackage{amsmath, amssymb}
\usepackage{hyperref}
\usepackage[margin=1in]{geometry}
\usepackage{booktabs}

% Metadata
\hypersetup{
    pdfauthor={Lucky Chin},
    pdftitle={The Measurement Interpretation of Black Hole Entropy: A Grazing Photon's Limited View}
}

\title{The Measurement Interpretation of Black Hole Entropy: A Grazing Photon's Limited View}
\author{Lucky Chin \\ \normalsize \texttt{LuckyBakhtin@protonmail.com}}
\date{\today}

\begin{document}

\maketitle

\begin{abstract}
We propose an operational interpretation of the Bekenstein-Hawking entropy $S_{\mathrm{BH}} = A/(4\ell_P^2)$ as the maximum information measurable by a null probe grazing the event horizon. Modeling the horizon's information content as Planck-scale curvature correlations, we posit that a photon can only distinguish holistic correlation patterns enumerated by the integer partition function $p(n)$. Equating the observable entropy $\ln p(n)$ to $S_{\mathrm{BH}}$ reveals that the probe can resolve patterns built from $n = k N^2$ correlation units, where $k = 3/(32\pi^2) \approx 0.009499$. Thus, the area law reflects not the total information content, but the finite fraction ($\sim 0.9499\%$) accessible before the photon is captured. The model yields a universal logarithmic correction with coefficient $-2$.
\end{abstract}

\section{Introduction: Entropy as Accessible Information}

The Bekenstein-Hawking entropy, $S_{\mathrm{BH}} = A/(4\ell_P^2)$, is widely regarded as counting the microstates of a black hole \cite{bekenstein1973, hawking1975}. Holography suggests these microstates reside on the horizon \cite{thooft1993, susskind1995}, but this raises a practical question: how much of this information can an external observer actually measure?

We reframe the problem in operational terms: rather than deriving $S_{\mathrm{BH}}$ from first principles, we ask what the entropy implies about the \emph{limitations of measurement} at the horizon. Consider a massless probe on a grazing null geodesic—a photon skimming the event horizon. This probe has a finite proper time to interact with the horizon before being captured. What can it learn?

We propose that the photon measures not individual Planck-scale degrees of freedom, but \emph{correlations} among them, encoded in the Weyl curvature. The number of distinguishable correlation patterns it can resolve defines the \emph{accessible entropy}. We show that equating this accessible entropy to $S_{\mathrm{BH}}$ reveals a fundamental limit: the photon can only sample about $0.9499\%$ of the total possible curvature correlations.

\section{The Model: What a Grazing Photon Measures}

\subsection{Horizon Information as Curvature Correlations}

Let the horizon be partitioned into $N = A/\ell_P^2$ Planck areas. The total information capacity is $N$ bits, but an external probe cannot resolve individual bits due to redshift and quantum indeterminacy. Instead, the probe senses \emph{tidal forces}—the Weyl curvature—which arise from correlations between horizon elements.

The number of possible pairwise correlations scales as $N^2$. These correlations are the fundamental "alphabet" of the holographic screen. The photon detects \emph{patterns} of these correlations, not the correlations themselves as labeled entities.

\subsection{The Measurement Limit}

A grazing photon has a finite interaction time before gravitational focusing pulls it across the horizon. This limits the number of independent correlations it can sample. If it attempts to resolve too many details, it is captured. Thus, only a fraction $k$ of the $N^2$ total correlations are accessible.

We model the photon's measurement as the distinction of \emph{holistic curvature patterns}, counted by the integer partition function $p(n)$, where $n$ is the number of accessible correlation units. Partitions are appropriate because:
\begin{itemize}
\item They count \emph{unlabeled} patterns—the units are indistinguishable.
\item They represent \emph{global configurations} rather than local excitations.
\end{itemize}

The measurable entropy is then:
\begin{equation}
S_{\mathrm{meas}} = \ln p(n).
\label{eq:meas-entropy}
\end{equation}

\section{Equating Measurable and Thermodynamic Entropy}

\subsection{The Hardy–Ramanujan Asymptotic}

For large $n$, the partition function satisfies \cite{hardy1918}:
\begin{equation}
p(n) \sim \frac{1}{4n\sqrt{3}} \exp\left( \pi \sqrt{\frac{2n}{3}} \right),
\quad
S_{\mathrm{meas}} \sim \pi \sqrt{\frac{2n}{3}} - \ln(4n\sqrt{3}).
\label{eq:HR}
\end{equation}

\subsection{The Bekenstein-Hawking Entropy as an Observational Limit}

We equate the measurable entropy to the thermodynamic entropy:
\[
S_{\mathrm{meas}} = S_{\mathrm{BH}} = \frac{N}{4}.
\]

Matching the leading-order term gives:
\[
\pi \sqrt{\frac{2n}{3}} = \frac{N}{4}
\quad\Rightarrow\quad
n = \frac{3}{32\pi^2} N^2.
\]

Thus, the fraction of total correlations accessible is:
\begin{equation}
k = \frac{3}{32\pi^2} \approx 0.009499.
\label{eq:k}
\end{equation}

\subsection{Interpretation}

The result $n = k N^2$ means that although the horizon has $\sim N^2$ possible correlation relationships, a grazing photon can only resolve those built from about $0.9499\%$ of them. The Bekenstein-Hawking entropy is not a count of all microstates, but of those \emph{distinguishable from outside} given finite measurement time.

\subsection{Logarithmic Correction}

Substituting $n$ back into the full asymptotic form yields:
\[
S_{\mathrm{meas}} \sim \frac{N}{4} - 2\ln N + \ln\left(\frac{8\pi^{2}}{3^{3/2}}\right) + \mathcal{O}(1/N),
\]
a universal logarithmic correction with coefficient $-2$ and a constant term $\ln(8\pi^{2}/3^{3/2}) \approx 2.72098$. This contrasts with other approaches to black hole entropy corrections \cite{cardy1986, page1993}.

\paragraph{Remark on $e$:} The natural exponential arises because the Hardy–Ramanujan formula contains $\exp(\pi\sqrt{2n/3})$. Inverting that relation and taking logarithms inevitably introduces $\ln$ terms. The tidy combination of $\ln 2$, $\ln 3$, and $\ln \pi$ into $\ln(8\pi^2/3^{3/2})$ shows why $e$ appears so naturally.

\section{Discussion}

\subsection{The Role of $k$ as an Information-Theoretic Veil}

The small value of $k$ reflects the horizon's role as a causal barrier. It quantifies the fraction of holographic data that can be non-destructively read by an external probe.

\subsection{Comparison with Quantum Gravity Models}

\begin{table}[h!]
\centering
\caption{Logarithmic correction coefficient $\alpha$ in $S \sim \frac{A}{4\ell_P^2} + \alpha \ln A$.}
\label{tab:comparison}
\begin{tabular}{lc}
\toprule
\textbf{Model} & $\alpha$ \\
\midrule
This work (measurement-based) & $-2$ \\
Loop Quantum Gravity & $-3/2$ \\
String Theory & $-1$ or $-3/2$ \\
Euclidean Path Integral & $-3/2$ \\
\bottomrule
\end{tabular}
\end{table}

The difference suggests that operational accessibility may impose stronger constraints than microstate counting alone \cite{wald2001}.

\section{Conclusion}

We have shown that the Bekenstein-Hawking entropy can be interpreted as the information accessible to a grazing photon, limited by horizon causality. The measurable correlation patterns are counted by partitions, yielding $n \approx 0.009499 N^2$ and a logarithmic correction of $-2$. This approach reframes black hole entropy in operational terms and highlights the role of measurement limits in thermodynamics \cite{boltzmann1877}.

\begin{thebibliography}{9}
\bibitem{bekenstein1973} Bekenstein, J. D. (1973). \textit{Physical Review D}, 7(8), 2333–2346.
\bibitem{hawking1975} Hawking, S. W. (1975). \textit{Communications in Mathematical Physics}, 43(3), 199–220.
\bibitem{boltzmann1877} Boltzmann, L. (1877). \textit{Über die Beziehung zwischen dem zweiten Hauptsatze der mechanischen Wärmetheorie und der Wahrscheinlichkeitsrechnung}. Wiener Berichte, 76, 373–435.
\bibitem{hardy1918} Hardy, G. H., and Ramanujan, S. (1918). \textit{Asymptotic formulae in combinatory analysis}. Proceedings of the London Mathematical Society, 17, 75–115.
\bibitem{thooft1993} 't Hooft, G. (1993). \textit{Dimensional reduction in quantum gravity}. In \textit{Salamfestschrift}, edited by A. Ali et al., World Scientific, 284–296.
\bibitem{susskind1995} Susskind, L. (1995). \textit{The world as a hologram}. Journal of Mathematical Physics, 36, 6377–6396.
\bibitem{wald2001} Wald, R. M. (2001). \textit{The thermodynamics of black holes}. Living Reviews in Relativity, 4, 6.
\bibitem{cardy1986} Cardy, J. L. (1986). \textit{Operator content of two-dimensional conformally invariant theories}. Nuclear Physics B, 270, 186–204.
\bibitem{page1993} Page, D. N. (1993). \textit{Information in black hole radiation}. Physical Review Letters, 71, 3743–3746.
\end{thebibliography}

\end{document}